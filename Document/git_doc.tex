\documentclass[12pt]{article}

\usepackage{hyperref}
\hypersetup{
    colorlinks,
    citecolor=black,
    filecolor=black,
    linkcolor=black,
    urlcolor=black
}

\usepackage{xcolor}
\definecolor{light-gray}{gray}{0.90}

\usepackage{listings}
\lstdefinestyle{Bash}
{	language=bash,
	%keywordstyle=\color{red},
	backgroundcolor=\color{light-gray},
	basicstyle=\ttfamily,
	xleftmargin=.25in,
	xrightmargin=.25in,
	breaklines=true,
	literate={\$}{{\textcolor{blue}{\$}}}1 
}


\begin{document}

\title{Git help}
\author{Justin}
\date{}
\maketitle

\section{Git}

Git is a piece of software for keeping tracks of version of code and other text files. It can be view/revert to old versions of code, diff between versions, make new branches of your code for testing, and can be used to aid code development among several people.

\section{Starting}

Git come installed on most Macs and Linux machines, but in case you want to installed an updated version via Macports.
\begin{lstlisting}[style=Bash]
sudo port install git-core +svn +doc +bash_completion +gitweb
\end{lstlisting}


Now with Git installed we can update some simple configuration info 
\begin{lstlisting}[style=Bash]
$ git config --global user.name "JT Erwin"
$ git config --global user.email jterwin@lpl.arizona.edu
\end{lstlisting}
Note that you will want to adapt the above to your name. I also added
\begin{lstlisting}[style=Bash]
$ git config --global core.editor amacs
$ git config --global merge.tool opendiff
\end{lstlisting}
Here I have set it up to use amacs (which is an alias to Aquamacs, a Mac GUI emacs) as the default editor as opposed to the default vim.

\subsection{First repository}
\begin{lstlisting}[style=Bash]
$ git init
\end{lstlisting}


\section{Branching}




\section{Using server}

We can set up the git repository on a server, and then we can clone, commit, push to it from multiple computers and or people.

\subsection{Setting up RSS key}


\subsection{Set up on Remote Server}
One the remote server we setup an empty git repository
\begin{lstlisting}[style=Bash]
$ cd ~/mycodes
$ mkdir project.git
$ cd project.git
$ git --bare init
\end{lstlisting}

The '--bare' is necessary because ?



\subsection{Clone to local computer}
Now on my personal computer I can clone that project. I put the git repository on HIPAS, and on my Macbook I have my shell set up so that the variable 'HIPAS=hipas.lpl.arizona.edu'.
\begin{lstlisting}[style=Bash]
$ cd ~/Document/work
$ git clone jterwin@$HIPAS:mycodes/project.git
$ cd project
\end{lstlisting}
The above clone, creates a folder named 'project' (as apposed to project.git), containing a clone of the git repository. 

When I do a commit, it is only to my local repository/branch. For others to see my commits, I need to push to the server
\begin{lstlisting}[style=Bash]
$ git commit -am 'fix for something'
$ git push origin master
\end{lstlisting}
Here 'master' is the name of the branch I am working on (the default is master), and 'origin' refers to the origin of the git repository (which you can check refers to the server location by looking at 'git config --list').



\end{document}
